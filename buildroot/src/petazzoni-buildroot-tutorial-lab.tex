\documentclass[a4paper,12pt,obeyspaces,spaces,hyphens]{article}
\usepackage[utf8]{inputenc}
\usepackage[T1]{fontenc}
\usepackage[normalem]{ulem}
\usepackage[vmargin=2cm, hmargin=1.5cm]{geometry}
\usepackage{fancyhdr}      % Used for customization of header/foter
\pagestyle{fancy}
\usepackage{graphicx}
\usepackage{titlesec}      % Used for \titleformat
\usepackage[htt]{hyphenat} % I want hyphenation in {\tt ...} blocks
\usepackage{xcolor}
\usepackage{titling}
\usepackage{eurosym}
\usepackage{ifthen}
\usepackage{hyperref}
\usepackage[all]{hypcap}
\usepackage{xparse}

\usepackage[most]{tcolorbox}
\definecolor{bg}{RGB}{220,220,220}

% A nicer font
\usepackage{palatino}

%% Remove indentation on the first line of each paragraph, and add
%% some space between each paragraph.
\usepackage{parskip}

\hypersetup{colorlinks=true,linkcolor=gray,urlcolor=gray}

\fancyhead{}
\fancyfoot{}

\fancyhead[CO, CE]{\includegraphics[width=\textwidth]{header.jpg}}
\fancyfoot[RO, LE] {\thepage}
\fancyfoot[LO, RE] {\small Embedded Apprentice Linux Engineer - Getting started with Buildroot - \url{https://bootlin.com}}

\renewcommand{\headrulewidth}{0pt}
\renewcommand{\footrulewidth}{0.5pt}

\setlength{\textheight}{56em}

\definecolor{textcolor}{HTML}{4B6FA9}

% Header style
\titleformat{\section}{\color{textcolor}\normalfont\Large\sffamily\bfseries}{}{1em}{}[\vspace{2pt}\titlerule]
\titleformat{\subsection}[display]{\normalfont\large\sffamily}{}{0pt}{}[\vspace{2pt}\titlerule]
\titleformat{\subsubsection}[display]{\normalfont\sffamily}{}{0pt}{}

% No section numbering
\setcounter{secnumdepth}{0}

% Customization of title and author style
\pretitle{\begin{center}\Huge\sffamily}
\preauthor{\begin{center}
\large \sffamily \lineskip 0.2em%
\begin{tabular}[t]{c}}
\postauthor{\end{tabular}\par\end{center}}
\predate{\begin{center}\large\sffamily}

% Useful definitions

\newcommand{\code}[1]
{\path{#1}}

\hypersetup{pdftitle={Getting started with Buildroot - Lab},
  pdfauthor={Thomas Petazzoni, Bootlin}}

\title{Getting started with Buildroot - Lab}
\author{Thomas Petazzoni, Bootlin}

\begin{document}

\maketitle
\thispagestyle{fancy}

These lab instructions are written for the {\em Getting started with
  Buildroot} tutorial of the {\em Embedded Apprentice Linux Engineer}
track. They are designed to work for the {\em PocketBeagle} hardware
platform.

\section{Initial configuration and build}

Our first step to get started with Buildroot is obviously to build a
minimal Linux system, with just a bootloader, Linux kernel and simple
user-space.

The Buildroot at \url{https://github.com/e-ale/buildroot-e-ale}
provides a ready-to-use {\em defconfig} to the PocketBeagle
platform. However, for educational purposes, we are going to start
straight from the official vanilla Buildroot, and build our
configuration from scratch.

\subsection{Getting Buildroot}

Start by cloning Buildroot's official Git repository:

\begin{verbatim}
$ git clone git://git.busybox.net/buildroot
$ cd buildroot/
\end{verbatim}

(Note: if download speed is too slow, you can use the
\code{buildroot.tar.xz} tarball provided by the instructor)

We'll base our work on the 2018.02 release, which is the latest LTS
release, so we create a branch based on the 2018.02 tag:

\begin{verbatim}
$ git checkout -b e-ale 2018.02
\end{verbatim}

\subsection{Creating a minimal configuration}

Now it's time to start creating our configuration:

\begin{verbatim}
$ make menuconfig
\end{verbatim}

In the configuration, we'll have to customize a number of options, as
detailed below. Of course, take this opportunity to navigate in all
the options, and discover what Buildroot can do.

\begin{itemize}

\item In {\em Target options}

  \begin{itemize}

  \item Change {\em Target architecture} to {\em ARM (little endian)}

  \item Change {\em Target architecture variant} to {\em Cortex-A8}

  \end{itemize}

\item In {\em Build options}, set {\em global patch directories} to
  \code{board/e-ale/pocketbeagle/patches/}. This will allow us to put
  patches for Linux, U-Boot other packages in subdirectories of
  \code{board/e-ale/pocketbeagle/patches/}.

\item In {\em Toolchain}

  \begin{itemize}

  \item Change {\em Toolchain type} to {\em External toolchain}. By
    default, Buildroot builds its own toolchain, but it can also use
    pre-built external toolchain. We'll use the latter, in order to
    save build time. Since we've selected an ARM platform, a Linaro
    toolchain is automatically selected, which will work for us.

  \end{itemize}

\item In {\em System configuration}, you can customize the {\em System
    host name} and {\em System banner} if you wish. Keep the default
  values for the rest.

\item In {\em Kernel}

  \begin{itemize}

  \item Enable the {\em Linux kernel}, obviously!

  \item Choose {\em Custom version} as the {\em Kernel version}

  \item Choose {\em 4.14.24} as {\em Kernel version}

  \item Patches will already be applied to the kernel, thanks to us
    having defined a {\em global patch directory} above.

  \item Choose \code{omap2plus} as the {\em Defconfig name}

  \item We'll need the Device Tree of the PocketBeagle, so enable {\em
      Build a Device Tree Blob (DTB)}

  \item And use \code{am335x-pocketbeagle} as the {\em Device Tree
      Source file names}

  \end{itemize}

\item In {\em Target packages}, we'll keep just Busybox enabled for
  now. In the next sections, we'll enable more packages.

\item In {\em Filesystem images}, enable {\em ext2/3/4 root
    filesystem}, select the {\em ext4} variant. You can also disable
  the {\em tar} filesystem image, which we won't need.

\item In {\em Bootloaders}, enable {\em U-Boot}, and in {\em U-Boot}:

  \begin{itemize}

  \item Switch the {\em Build system} option to {\em Kconfig}: we are
    going to use a modern U-Boot, so let's take advantage of its
    modern build system!

  \item Use a {\em Custom version} of value {\em 2018.01}. You'll
    notice that the current default is already {\em 2018.01}. However,
    Buildroot upstream is regularly updating this to the latest U-Boot
    version. However, to have a reproducible setup, we really want to
    use a fixed version.

  \item Use \code{am335x_pocketbeagle} as the {\em Board defconfig}

  \item The {\em U-Boot binary format} should be changed from
    \code{u-boot.bin} to \code{u-boot.img}. Indeed, this second stage
    bootloader will be loaded by a first stage bootloader, and needs
    to have the proper header to be loaded by the first stage.

  \item Enable {\em Install U-Boot SPL binary image} to also install
    the first stage bootloader. Its name in {\em U-Boot SPL/TPL binary
      image name(s)} should be changed to \code{MLO} since that's how
    U-Boot names it, and how the AM335x expects it to be named.

  \end{itemize}

\end{itemize}

As you have noticed, in the configuration, we have referenced
\code{board/e-ale/pocketbeagle/patches} as a directory containing
patches for various packages. We now need to add the U-Boot and Linux
patches that add support for the {\em PocketBeagle}, which are not
upstream yet. They are available from the \code{patches/} folder in
the USB stick provided by your instructor, just copy it to
\code{board/e-ale/pocketbeagle} so that you get the following
directory hierarchy:

\begin{tcolorbox}[enhanced jigsaw,colback=bg,boxrule=0pt,arc=0pt]
\begin{verbatim}
$ tree board/e-ale/pocketbeagle/
board/e-ale/pocketbeagle/
- patches
  - linux
    - 0001-Stripped-back-pocketbeagle-devicetree.patch
  - uboot
    - 0001-am335x_evm-uEnv.txt-bootz-n-fixes.patch
    - 0002-U-Boot-BeagleBone-Cape-Manager.patch
    - 0003-pocketbeagle-tweaks.patch
\end{verbatim}
\end{tcolorbox}

\subsection{Running the build}

To start the build, you can run just \code{make}. But it's often
convenient to keep the build output in a log file, so you can do:

\begin{verbatim}
$ make 2>&1 | tee build.log
\end{verbatim}

or alternatively use a wrapper script provided by Buildroot:

\begin{verbatim}
$ ./utils/brmake
\end{verbatim}

The build will take a while (about 14-15 minutes on your instructor's
machine). If the Internet connectivity is slow, the USB stick that
your instructor has provided contains the files to be downloaded. Just
copy them into the \code{dl/} folder, and restart the build.

The overall build takes quite some time, because the Linux kernel
configuration \code{omap2plus_defconfig}, which supports all OMAP2,
OMAP3, OMAP4 and AM335x platforms has a {\em lot} of drivers and
options enabled. It would definitely be possible to make a smaller
kernel configuration for the {\em Pocket Beagle}, reducing the kernel
size and boot time.

At the end of the build, the output is located in
\code{output/images}. We have:

\begin{itemize}
\item \code{MLO}, the first stage bootloader
\item \code{u-boot.img}, the second stage bootloader
\item \code{zImage}, the Linux kernel image
\item \code{am335x-pocketbeagle.dtb}, the Linux kernel Device Tree Blob
\item \code{rootfs.ext4}, the root filesystem image
\end{itemize}

However, that doesn't immediately give us a bootable SD card image. We
could create it manually, but that wouldn't be really nice. So move on
to the next section to see how Buildroot can create the SD card image
for you.

\subsection{Creating a SD card image}

To create a SD card image, we'll use a tool called \code{genimage},
which provided a configuration file, will output the image of a block
device, with multiple partitions, each containing a filesystem. See
\url{https://git.pengutronix.de/cgit/genimage/tree/README.rst} for
some documentation about {\em genimage} and its configuration file
format.

\code{genimage} needs to be called at the very end of the build. To
achieve this, Buildroot provides a mechanism called {\em post-image
  scripts}, which are arbitrary scripts called at the end of the
build. We will use it to create a SD card image with:

\begin{itemize}
\item A FAT partition containing the bootloader images, the kernel
  image and Device Tree
\item An ext4 partition containing the root filesystem
\end{itemize}

In addition, the U-Boot bootloader for the {\em PocketBeagle} is
configured by default to load a file called \code{uEnv.txt} to
indicate what should be done at boot time. This file should also be
stored in the first partition of the SD card.

So, go back to \code{make menuconfig}, and adjust the following
options:

\begin{itemize}

\item In {\em System configuration}
  \begin{itemize}
  \item Set {\em Custom scripts to run after creating filesystem
      images} to \code{board/e-ale/pocketbeagle/post-image.sh}
  \end{itemize}

\item In {\em Host utilities}, enable \code{host genimage},
  \code{host mtools} and \code{host dosfstools}. {\em mtools} and {\em
    dosfstools} are needed because our {\em genimage} configuration
  includes the creation of a FAT partition.

\end{itemize}

Copy \code{genimage.cfg}, \code{post-image.sh} and \code{uEnv.txt}
from the USB stick provided by your instructor to
\code{board/e-ale/pocketbeagle}.

Restart the build again. Once the build is finished, you should now
have a \code{sdcard.img} file in \code{output/images/}, ready for
flashing:

\begin{verbatim}
$ sudo dd if=output/images/sdcard.img of=/dev/mmcblk0 bs=1M
\end{verbatim}

Insert the SD card in the PocketBeagle, and boot it. Provided you have
a serial port connection at 115200 bps, you should see the system
booting. You can login as \code{root} with no password, and explore
the (very minimal) system. The system weights 18.4 MB, of which 12.1
MB are kernel modules.

\subsection{Storing our Buildroot configuration}

Our Buildroot configuration is currently stored as \code{.config},
which is not under version control and would be removed by a
\code{make distclean}. So, let's store it as a {\em defconfig} file:

\begin{verbatim}
$ make BR2_DEFCONFIG=configs/eale_pocketbeagle_defconfig savedefconfig
\end{verbatim}

And then look at \code{configs/eale_pocketbeagle_defconfig} to see
what your configuration looks like.

\section{Network over USB and SSH connection}

In this section, we will setup network over USB and add a SSH server
to our build.

To enable network over USB, we need to change two things to our root
filesystem:

\begin{itemize}

\item Add an init script that will load the necessary kernel modules,
  and do the appropriate \code{configfs} tweaks to set up the USB
  gadget device. This script will be installed in
  \code{/etc/init.d/S30usbgadget} in the root filesystem.

\item Customize the \code{/etc/network/interfaces} file to set up the
  \code{usb0} network interface.

\end{itemize}

In order to achieve this, we are going to use the concept of {\em root
  filesystem overlay} of Buildroot. Filesystem overlays are simply
directories that get copied over the root filesystem at the end of the
build.

Go to the Buildroot configuration, and in {\em System configuration},
change {\em Root filesystem overlay directories} to
\code{board/e-ale/pocketbeagle/overlay/}. In the same menu set the
{\em Root password} to a non-empty string of your choice. Finally in
the {\em Target packages} menu, {\em Networking applications} submenu,
enable \code{dropbear}.

Then, copy the contents of the \code{overlay/} directory available on
the USB stick provided by your instructor to
\code{board/e-ale/pocketbeagle/overlay/}. Take this opportunity to
have a look at the contents of the overlay.

Restart the build, reflash your system, and boot it. You should see
the USB Gadget being configured at boot time. Its IP address is
\code{192.168.42.2}. On your PC, assign \code{192.168.42.1} as the IP
address of the USB network interface that just appeared, and then you
can connect over SSH to your PocketBeagle:

\begin{verbatim}
$ ssh root@192.168.42.2
\end{verbatim}

\section{Developing an application}

\subsection{Adding and testing {\em libgpiod}}

Now, we will improve our application to make it use GPIOs through the
{\em gpiod} library. To achieve this:

\begin{itemize}

\item Enable \code{libgpiod} and its tools from Buildroot
  \code{menuconfig}

\item Enable the \code{CONFIG_PINCTRL_MCP23S08} option in the Linux
  kernel configuration, by running \code{make linux-menuconfig}. This
  option enables the driver of the GPIO expander used on the {\em
    BaconBits} add-on board. Note that this change to the Linux kernel
  configuration will be lost if you do a \code{make clean}. To make
  things persistent, the nicest solution is to add
  \code{CONFIG_PINCTRL_MCP23S08=y} to a new config fragment file in
  \code{board/e-ale/pocketbeagle/} and reference this fragment from
  \code{BR2_LINUX_KERNEL_CONFIG_FRAGMENT_FILES} Buildroot option. This
  way your Linux kernel configuration is the standard
  \code{omap2plus_defconfig} plus your tweaks.

\end{itemize}

Re-build your system and reflash it. You should now be able to play
with GPIOs with the {\em libgpiod} tools. The syntax of the
\code{gpioset} program is \code{gpioset <chip name/number>
  <offset1>=<value1>}:

\begin{tcolorbox}[enhanced jigsaw,colback=bg,boxrule=0pt,arc=0pt]
\begin{verbatim}
# gpioset gpiochip4 0=0
# gpioset gpiochip4 1=0
# gpioset gpiochip4 1=1
# gpioset gpiochip4 0=1
\end{verbatim}
\end{tcolorbox}

\subsection{Cross-compiling a simple application}

The USB stick provided by your instructor contains a small application
that uses {\em libgpiod}, in the \code{eale-gpio-app} folder. Copy
this folder side by side to Buildroot:

\begin{tcolorbox}[enhanced jigsaw,colback=bg,boxrule=0pt,arc=0pt]
\begin{verbatim}
- buildroot
  - arch
  - output
  - ...
- eale-gpio-app
  - eale-gpio-app.c
  - Makefile
\end{verbatim}
\end{tcolorbox}

For now, we'll build it manually:

\begin{verbatim}
$ ./output/host/bin/arm-linux-gnueabihf-gcc -o eale-gpio-app \
    ../eale-gpio-app/eale-gpio-app.c -lgpiod
\end{verbatim}

Transfer it to the target:

\begin{verbatim}
$ scp eale-gpio-app root@192.168.42.2:
\end{verbatim}

And then run it on the target:

\begin{verbatim}
# ./eale-gpio-app
\end{verbatim}

\subsection{A package for our application}

Compiling your application manually is fine during development and if
your application is simple, but it would be better to have Buildroot
build it for us. To achieve this, we need to create a new Buildroot
package.

Start by creating the \code{package/eale-gpio-app} directory. Inside
this directory, place a \code{Config.in} file containing:

\begin{tcolorbox}[enhanced jigsaw,colback=bg,boxrule=0pt,arc=0pt]
{\footnotesize
\begin{verbatim}
config BR2_PACKAGE_EALE_GPIO_APP
        bool "eale-gpio-app"
        depends on BR2_TOOLCHAIN_HEADERS_AT_LEAST_4_8 # libgpiod
        select BR2_PACKAGE_LIBGPIOD
        help
          This is the E-ALE GPIO demo application.

comment "eale-gpio-app needs kernel headers >= 4.8"
        depends on !BR2_TOOLCHAIN_HEADERS_AT_LEAST_4_8
\end{verbatim}
}
\end{tcolorbox}

What this file does is:

\begin{itemize}

\item Describe an option that will be visible in {\em menuconfig} to
  enable/disable the compilation of our demo application

\item {\em Selects} the libgpiod package, because it is a dependency
  of our demo application.

\item The \code{BR2_TOOLCHAIN_HEADERS_AT_LEAST_4_8} dependency is
  inherited from {\em libgpiod}. {\em libgpiod} requires at least
  kernel headers from Linux 4.8, and therefore our application also
  has this requirement.

\end{itemize}

Then, edit \code{package/Config.in} and add a new line in the {\em
  Hardware handling} category to include the newly created
\code{package/eale-gpio-app/Config.in}.

Now, we need to describe how to build this package, by adding
\code{package/eale-gpio-app/eale-gpio-app.mk} with the following
contents:

\begin{tcolorbox}[enhanced jigsaw,colback=bg,boxrule=0pt,arc=0pt]
{\footnotesize
\begin{verbatim}
################################################################################
#
# eale-gpio-app
#
################################################################################

EALE_GPIO_APP_SITE = $(TOPDIR)/../eale-gpio-app
EALE_GPIO_APP_SITE_METHOD = local
EALE_GPIO_APP_DEPENDENCIES = libgpiod

define EALE_GPIO_APP_BUILD_CMDS
    $(TARGET_MAKE_ENV) $(MAKE) $(TARGET_CONFIGURE_OPTS) -C $(@D)
endef

define EALE_GPIO_APP_INSTALL_TARGET_CMDS
    $(TARGET_MAKE_ENV) $(MAKE) $(TARGET_CONFIGURE_OPTS) -C $(@D) \
        DESTDIR=$(TARGET_DIR) install
endef

$(eval $(generic-package))
\end{verbatim}
}
\end{tcolorbox}

Make sure that the lines inside \code{EALE_GPIO_APP_BUILD_CMDS} and
\code{EALE_GPIO_APP_INSTALL_TARGET_CMDS} are prefixed with one tab.

What this file does is:

\begin{itemize}

\item Describe where the source code for our application is
  located. Here, we use the {\em local} site method, to indicate that
  the source code is locally available (as opposed to a tarball
  available on a HTTP server)

\item Describe the application dependency. This ensures that {\em
    libgpiod} gets built before our application.

\item Describe how to build and install our application, by using the
  {\em Makefile} provided by the application. \code{TARGET_MAKE_ENV}
  and \code{TARGET_CONFIGURE_OPTS} are variables provided by
  Buildroot, which define standard variables such as \code{CC},
  \code{CFLAGS}, etc. to make sure the application gets
  cross-compiled.

\end{itemize}

With this in place, go to {\em menuconfig} and enable your
application. Restart the build with \code{make}. You should now see
your application being built and installed to the root
filesystem. Reflash the SD card image, and verify that your
application is in \code{/usr/bin/} on the target, and that it works as
expected. If it does, congratulations!

\section{Going further}

\subsection{Using Linux kernel configuration fragment}

When we enabled support for the MCP23S08 GPIO expander, we kept the
Linux kernel configuration change only in
\code{output/build/linux-4.14.24}, which means it will be lost when we
will clean the build. To fix this:

\begin{itemize}

\item Add \code{CONFIG_PINCTRL_MCP23S08=y} to a new file called
  \code{board/e-ale/pocketbeagle/linux.frag}

\item In {\em menuconfig}, set
  \code{BR2_LINUX_KERNEL_CONFIG_FRAGMENT_FILES} to
  \code{board/e-ale/pocketbeagle/linux.frag}

\end{itemize}

Then, do a complete rebuild of your kernel: \code{make linux-dirclean}
followed by \code{make}.

\subsection{Generate legal information}

To help with license compliance, Buildroot can generate a legal
report:

\begin{verbatim}
$ make legal-info
\end{verbatim}

Then look in \code{output/legal-info}, where you will find the license
text for all packages, tarballs, as well as a manifest in the form of
a CSV file (\code{manifest.csv}).

\subsection{Analyzing the build}

Sometimes, it is useful to analyze the size of the filesystem or the
build duration. Buildroot provides convenient tooling to achieve
this. For the data to be correct, we need to have a completely clean
build:

\begin{verbatim}
$ make clean all
\end{verbatim}

Once the build is done, run \code{make graph-size} and look at
\code{output/graphs/graph-size.pdf}. Then run \code{make graph-build}
and look at \code{output/graphs/build.hist-build.pdf}.

\end{document}
